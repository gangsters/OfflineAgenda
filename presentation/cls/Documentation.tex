\documentclass[asi,compacte,brouillon]{picINSA}

\version{1.00}
\referenceVersion{Documentation}
\title{Documentation package INSA}
\author{Pierre Sassoulas}

\begin{document}
\maketitle{}
\tableofcontents{}
\chapter*{Introduction}

Ce document permet d'utiliser plus facilement le package picINSA.cls. Il détaillera d'abord les options proposées, puis les différents environnements au niveau de la page de service et du document complet.

\chapter{Options possibles}

\section{Département}

Il peut être \textbf{asi} ou \textbf{mrie}. Cela change la présentation général du document et le contenu de la commande \verb+\departement{}+

\section{Brouillon}

Il peut être utile de signaler qu'une version est un brouillon et pas la version définitive. Pour cela on utilise l'option \textbf{brouillon} qui ajoute une watermark sur chaque page.

\section{Compacte}

Par défaut les document sont formaté de manière à mieux présenter quand il sont imprimé. Les chapitres commencent toujours sur la page de droite, la page de service également\dots{}
Ce n'est pas l'idéal si vous comptez seulement lire le fichier sur un ordinateur en .pdf.

L'option \textbf{compacte} supprime ses pages blanches.

 \section{Options régissant le bas de page}
\subsection{Page, Version}

Par défaut on affiche le numéro de page et la version du document. Pour ne pas les afficher, vous pouvez utiliser les options \textbf{sansPage} et \textbf{sansVersion}.

Le numéro de version est initialisé avec la commande \verb+\version{numero_version}+

\subsection{Numero, Date, Nom}

Dans le cas ou votre document est numéroté, daté ou nommée, vous pouvez utilisez une option parmis \textbf{numero}, \textbf{date} et \textbf{nom}. Par défaut, cela va supprimer l'affichage de la version. En utilisant \textbf{sansPage} vous pouvez afficher la version en plus de votre première option, ou simplement votre première option si vous utilisez \textbf{sansVersion}.

Les options sont :
\begin{itemize}
\item \textbf{numero} qui est initialisé avec la commande \verb+\numero{numero}+
\item \textbf{date} qui est initialisé avec la commande \verb+\dateDocument{2013-03-28 (ou \today{})}+
\item \textbf{nom} qui est initialisé avec la commande \verb+\nom{nom_document}+
\end{itemize}


\chapter{Environnements de la page de service}

\chapter{Environnements à l'intérieur des documents}

\end{document}
